\documentclass{article}
\title{Global Script Syntax Overview, Part 1 (Syntax.hsgs)}
\author{Jonathan Cast\\$<$\texttt{jonathanccast\@fastmail.fm}$>$}

\usepackage{haskell}

\begin{document}

For this video, I'm going to go through some actual Global Script code and point out significant features of the syntax.

Disclaimer up-front:
This is based on my first cut of the language.
All of this syntax is implemented and does work,
but my second implementation, in progress, will change some of the details.
Also, it's a rough implementation, so a number of features I do plan to implement are missing,
such as operator precedence.

Explain basic structure of the file, so people know what they're looking at.

Or - consider this - clean up the file so it's a little more like real Global Script?

There's no type-checker,
so any function that should be a type-class method has to be replaced with a qualified function.
But it's close enough to Global Script to point out some interesting ideas.

This file is the parser for the new Global Script implementation.

Point out Unicode λ character.

Single quotes for variables being defined.
Any variable that's being defined, anywhere in the language, has a single quote in front of it.
Then you remove that single quote when you use the variable.
This solves the problem of telling variables from constructor names in patterns;
it's also compatible with having views in the language, which Global Script does,
and it helps with ambiguities in a couple of other places.

Point out periods for lambda-like functions.

Periods don't interfere with qualified names because qualified names cannot contain whitespace.

Unary operators.

Lambda-like functions are just library functions.

Eventually, we'll have a for function with a type-class-qualified type,
but since there's not type-checker, we have to use a qualified function for now.

Of course, being able to use a qualified function for something like for is a feature.

You have to modify the hsgs2hs program to define a new foo.for function now,
in order to specify how the ← operator will work.
The type-class based for function will eventually remove the need for that.

Eventually, you'll be able to define third-party functions with a syntax similar to for in a library.

No way to use just an expression as a generator in a for.
We could allow it, but it's not a syntax I'm fond of.
Use \<*>\> instead or use \<\_ ←\>.

You can use - or . to separate words.

No precedence, so \<*>\> isn't higher precedence than \<<|>\>.
When precedence is implemented, the parentheses will go away.
Also eventually we'll support : for over-riding precedence based on layout.

String syntax.
Copied from Perl 5.
Several different kinds of strings are supported;
major ones are;
qq{} - machine readable string
log{} - human readable markup string
r{} - single character

qq{} - program logic based on this
log{} - only read by a human being (or AI)

unit is a better term than return or pure.

; is required when there are multiple generators but eventually not when there's only one.

\<\#f\> field / method extraction syntax.
\<\#0\>, \<\#1\>, etc. for tuples.
Single quotes are useful here - we know the next thing is a variable name so we're not surprised.
Tuples are literally just records whose fields are named 0, 1, 2, etc.
Extensible tuples.

analyze/case for case analysis.
analyze is just a library function, with two arguments.
case is also just a library function,
but it takes a pattern as the first argument,
and it has an else clause argument,
which is normally another call to case,
and is undefined if you omit it.

Discuss how insufficient case errors work.
analyze with no cases.
First case (after case nothing).
Point out the case below that (in expr.function.w) with \<order-of-being.pattern : 's1\>,
where the head will be evaluated in the missing case error.

Call out the ← unit pattern and show off ∝.

Qualified constructor names.

Record literal syntax.

Monomorphism restriction based on ∝ vs =.

\<\&\&\> vs \<∧\>

list.∈ will short-circuit

log.dstr;
gsvar.fmt-atom should return a log.t,
but doesn't because it's written in Haskell

\end{document}
